
%%%%%%%%%%%%%%%%%%%%%%%%%%%%%%%%%%%%%%%%%%%%%%%%%%%%%%%%%%%%%%%
% EDITORIAL SECTION
%
\documentclass{PSAIE}%
\begin{document}%
\PSAIEHeadFirst{10}{1}{1}{3}%
%
% END OF EDITORIAL SECTION
%%%%%%%%%%%%%%%%%%%%%%%%%%%%%%%%%%%%%%%%%%%%%%%%%%%%%%%%%%%%%%%%


% Please give a short title for the running head
\fancyhead[CO]{\PSAIEheader{Temporary title}} % TODO: Make actual title
% Short list of authors for the running head
\fancyhead[CE]{\PSAIEheader{L. Kov\'acs and A. Ag\'ardi}}
\fancyfoot{}

\noindent\PSAIEtitle{Temporary title} % TODO: Make actual title

\noindent\PSAIEauthor{Bal\'azs Bolyki}
{University of Miskolc, Hungary\\[0pt] Informatics Institute}
{bolyki@iit.uni-miskolc.hu}

\noindent\PSAIEauthor{L\'aszl\'o \'Arvai}
{University of Miskolc, Hungary\\[0pt] Informatics Institute}
{arvai.laszlo@iit.uni-miskolc.hu}

\noindent\PSAIEauthor{Dr. Szilvia \'Arvai-Homolya}
{University of Miskolc, Hungary\\[0pt] Mathematics Institute}
{mathszil@uni-miskolc.hu}

\noindent\PSAIEreceived{\today}

\noindent\PSAIEabstract{This paper is a template for those authors
    who wish to prepare their manuscript to be published in
    \emph{Production Systems and Information Engineering} by using the
    \emph{amsart} document class. You can reedit the text of this
    paper and the corresponding \emph{bib} file in order to obtain
    your manuscript.}

\noindent\PSAIEkey{dataset, grapevine}

%%%%%%%%%%%%%%%%%%%%%%%%%%%%%%%%%%%%%%%%%%%%%%%%%%%%%%%%%%%%%%%

\section{Introductions}
Robotization of grapevine pruning has been becoming an area of interest in agriculture recently
\cite{botterill2017robot}, \cite{fernandes2021grapevine}, \cite{katyara2020reproducible}. The reason
comes down to the fact, that proper pruning requires expertise, and it is a time consuming activity,
resulting in expensive human workforce. Also, tools for executing such a task are now available.

One of the subtasks of grapevine pruning is reconstruction of the plant, based on images made by
one or multiple cameras, mounted on the robot. After plant reconstruction, the system can make decisions
as to where pruning has to be performed. Convolutional neural networks are available to do
object detection and localization on images \cite{glenn_jocher_2021_5563715}, \cite{matterport_maskrcnn_2017},
\cite{liu2016ssd}. However, such approaches require great amounts of training data. Creating a training
dataset of sufficient size and quality is laborous, and topic specific datasets are usually not available.

In this paper, we are focusing on generating and extending an existing, topic specific dataset, namely
a bounding box based dataset created on grapevines. The existing dataset does not provide the information
required for determining pruning points on the grapevine. Consequently, the training data needs to be
extended. In later sections we propose a methodology for generating a new dataset algorithmically, based on
the original dataset, using area specific knowledge on the grapevine, while requiring as little human
intervention as possible.

\section{Task description} \label{sec_task_description}

\section*{Acknowledgements}
\noindent
The authors express their gratitude to the XXX Institute at YYY for
their hospitality, etc.

\bibliographystyle{PSAIEbib}
\bibliography{citations}

\end{document}